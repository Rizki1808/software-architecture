\chapter{Pendahuluan}
\author{Hendra Lijaya, Oktavianus Hendry Wijaya}

\section{Pengertian}
DevOps merupakan metode pengembangan software dengan mengkolaborasikan \textit{software developer} dengan \textit{IT operation}. 
\section{Fungsi}
Tujuan akhir atau \textit{goal} dari DevOps adalah untuk menciptakan lingkungan kolaborasi yang berkelanjutan untuk membawa software menjadi lebih berkualitas, lebih cepat, dan dapat diandalkan.
\section{Kelebihan Kekurangan}
Berikut adalah kelebihan dan kekurangan arsitektur DevOps:

\subsection{Kelebihan}
Keuntungan dari menerapkan arsitektur DevOps adalah:
\begin{itemize}
	\item DevOps menjadi pilihan yang bagus untuk \textit{development} dan \textit{deployment} aplikasi yang cepat
	\item Merespon lebih cepat ke perubahan market untuk meningkatkan \textit{business growth} (pertumbuhan bisnis)
	\item Data dapat disentralisasikan sehingga meningkatkan konsistensi data dan mengurangi duplikasi data.
	\item  DevOps meningkatkan profit bisnis dengan mengurangi waktu \textit{delivery software} dan biaya.
	\item DevOps menghilangkan proses deskriptif sehingga memberikan kejelasan mengenai \textit{development} dan \textit{delivery product}.
	\item Meningkatkan \textit{experience} dan kepuasan \textit{customer}.
	\item DevOps menyederhanakan kolaborasi dan menggunakan semua \textit{tools} di cloud untuk diakses pengguna.
	\item Meningkatkan keterlibatan dan produktivitas tim.
\end{itemize}
\subsection{Kekurangan}
Kekurangan dari penerapan arsitektur DevOps adalah:
\begin{itemize}
	\item DevOps \textit{professional} atau \textit{expert} masih belum umum ditemukan.
	\item \textit{Developing} dengan DevOps mahal.
	\item Penerapan DevOps baru ke dalam industri sulit untuk dikelola dalam waktu singkat.
	\item 	Kurangnya pengetahuan mengenai DevOps dapat menyebabkan masalah pada \textit{Continuous Integration} dari \textit{project automation}.
\end{itemize}
\section{Perbedaan DevOps dan nonDevOps}
Perbedaan besar dari \textit{development} dengan DevOps dan nonDevOps yaitu:
	\begin{enumerate}
		\item Kolaborasi
		\\Pada \textit{Software development} nonDevOps, developer dan tim operation bekerja secara terpisah. Sedangkan pada DevOps, kedua hal tersebut bekerja secara kolaboratif dalam 2 tim yang berbeda dengan berbagi pengentahuan dan skill sehingga memastikan proses software development dapat disederhanakan.
		
		\item \textit{Continuous Integration and Delivery (CI/CD)}
		\\DevOps menerapkan CI/CD yang dimana melibatkan sistem \textit{automation} pada proses \textit{software development} mulai dari \textit{building} dan \textit{testing} hingga ke \textit{deployment} dan \textit{maintenance}. Dengan menerapkan ini, perubahan dapat di tes dan diintegrasikan ke software secara cepat dan efisien mungkin.
		\item \textit{Automation}
		\\DevOps bergantung secara penuh pada automation untuk meningkatkan efisiensi dan mengurangi error. Tools seperti \textit{management configuration}, \textit{continuous integration}, dan \textit{continuous delivery} memungkinkan tim untuk mengautomatis proses yang awalnya manual dan memastikan konsistensi.
		\item \textit{Monitoring}
		\\Tim yang menerapkan DevOps menggunakan \textit{tools} untuk \textit{monitoring} dan analitik untuk mengumpulkan data mengenai performa pada software saat \textit{production}. Hal ini membantu tim dalam mengidentifikasi dan menyelesaikan isu dengan cepat sehingga mengurangi downtime dan meningkatkan \textit{user experience} secara keseluruhan.
		\item Agile Development
		\\DevOps berdasarkan pada prinsip \textit{agile development} yang dimana fleksibilitas, adaptabilitas, dan kolaborasi sangat ditekankan. Tim DevOps memprioritaskan dalam \textit{delivery} dalam perubahan kecil dan perubahan \textit{incremental} dengan cepat dibandingkan perilisan monolitik yang bersifat besar.
	\end{enumerate}
	Kesimpulannya adalah DevOps bersifat lebih kolaboratif, \textit{automated}, dan \textit{agile} pada proses\textit{ software development} yang menekankan \textit{continuous integration and delivery, automation, }dan \textit{monitoring}.

\section{Tools}
Tools yang digunakan dalam pembuatan DevOps:
\begin{enumerate}
	\item Git - GitHub Action
	\\GitHub Action adalah fitur dari \textit{platform} GitHub yang memungkinkan developer untuk mengautomasi \textit{workflows} dan \textit{build, test}, dan \textit{deploy} kode langsung  dari \textit{platform} GitHub.
	GitHub Action menyediakan \textit{library} dari \textit{pre-built actions }yang dapat digunakan untuk membangun \textit{workflows} dan juga kemampuan untuk membuat action kustom menggunakan JavaScripts atau Docker containers. \textit{Workflows} dapat dipicu/di\textit{trigger} oleh \textit{events} seperti push kode, request pull, atau pembuatan perilisan baru.
	\\Keuntungan menggunakan GitHub:
	\begin{itemize}
		\item Terintegrasi dengan GitHub
		\item Workflows yang dapat dikustomisasi
		\item Reusability
		\item Kolaborasi
		\item Skalabilitas
		\item Gratis
	\end{itemize}
	Kesimpulan, GitHub Action merupakan \textit{tools} yang sangat berguna untuk \textit{automating software development workflows}, menyediakan developer fleksibilitasm, kustomisasi, dan platform yang terintegrasi untuk \textit{building, testing,} dan \textit{deploy} kode
	
	\item Heroku
	Heroku merupakan \textit{platform cloud} yang memungkinkan developer untuk \textit{build, deploy}, dan mengelola aplikasi secara cepat dan mudah. Heroku mendukung beberapa Bahasa pemrograman seperti Java, Ruby, Node.js, Python, PHP, dan Go. Heroku menyediakan platform yang dikelola secara penuh sehingga developer tidak perlu mengkhawatirkan mengenai mengelola infrastruktur, sistem operasi, dan server.
	\\Heroku didasarkan pada arsitektur yang berbasis \textit{container} dan menggunakan Dyno untuk menjalankan aplikasi. Dyno merupakan \textit{container} linux yang ringan dan terisolasi yang berjalan diatas platform Heroku. Dyno di\textit{design} untuk menjalankan satu proses atau layanan yang membantu meningkatkan performa, skalabilitas, dan ketahanan.
	\\Fitur-fitur Heroku:
	\begin{itemize}
		\item \textit{Command Line Interface }(CLI)
		\item \textit{Web Based Dashboard}
		\item Beragam \textit{add-ons} dan \textit{extensions}
		\item\textit{ Support continuous integration and continuous delivery (CI/CD) workflows}.
	\end{itemize}
	Adapun kekurangan dari Heroku yaitu:
	\begin{itemize}
		\item Kustomisasi yang terbatas.
		\item Bergantung pada \textit{add-on third party}.
		\item Memerlukan biaya dan kartu kredit.
	\end{itemize}
\end{enumerate}
\section{Contoh Kasus}
\section{Code}
